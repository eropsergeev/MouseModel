\documentclass{article}
\usepackage [utf8x] {inputenc}
\usepackage [T2A] {fontenc}
\usepackage{amsmath}
\DeclareMathOperator{\tg}{tg}
\newcommand{\tgx}{\tg x}
\usepackage[pdftex]{graphicx}
\graphicspath{{pics/}}
\DeclareGraphicsExtensions{.pdf,.png,.jpeg}

\begin{document}

Классы полей:

\begin{itemize}

\item \textit{AbstractField}
    
    Базовый класс поля.

    Методы:
    \begin{itemize}
        \item \textit{Конструктор}

        \item \textit{get\_max\_dist}: в наследниках класса должен возвращать максимальное расстояние, которое
может быть пройдено по прямой на данном поляе.

        \item \textit{move}: принимает координаты точки и вектор, на который движется моделируемое
животное. Возвращает координаты точки, где остановилось животное (с учетом препятствий).

    \end{itemize}

\item \textit{CircleField}
    
    Встроенная реализация поля. Представляет из себя круг с круглыми же
препятствиями.

    Методы:
    \begin{itemize}
        \item \textit{Конструктор}: принимает радиус поля и набор препятствий. Дополнительно может принимать выход, являющийся реализацией \textit{Shape} (см. ниже).

        \item Остальные являются реализацией соответствующих методов \textit{AbstractField}.

    \end{itemize}


\end{itemize}

Классы моделей:

\begin{itemize}
    \item \textit{AbstractModel}

    Базовый класс для модели животного.

    Методы:
    \begin{itemize}
        \item \textit{Конструктор}: принимает поле и координаты животного на этом поле.

        \item \textit{move}: производит перемещение животного по полю в соответствии с моделью.

        \item \textit{get\_neurons}: возвращает вектор значений, соответствующий активностям нейронов
        животного в заданном месте
    \end{itemize}

    \item \textit{RandomModel}

    Тривиальная модель, основанная на случайном блуждании.

    Методы:

    \begin{itemize}
        \item \textit{Конструктор}: кроме поля и координат принимает шаг, на который будет перемещаться
животное, покрытие поля некоторыми областями и количество нейронов, которое необходимо
возвращать.

    \item \textit{move}: делает попытку переместить животное на заданный в конструкторе шаг в
случайном направлении.

    \item \textit{get\_neurons}: возвращает вектор из заданного в конструкторе числа нейронов, состоящий
из значений, соответствующих нейронам места и случайных значений.
    \end{itemize}

    \item \textit{NNModel}

    Модель на основе нейронной сети.

    Методы:

    \begin{itemize}
        \item \textit{Конструктор}: кроме поля и координат принимает собственно нейронную сеть,
способную обучаться методом Q-learning, а так же функции для преобразования данных на входе
и выходе сети и вспомогательные параметры.

    Список доп параметров:

    \begin{itemize}
        \item \textit{state\_func} -- принимает поле и координаты на нем, возвращает вход нейронной сети.

        \item \textit{move\_func} -- принимает сеть и ее вход, возвращает ход (пару коорбинат).

         \item \textit{reward\_func} -- принимает поле, координаты стартовой и конечной точки, возвращает награду сети (число).

         \item \textit{empty\_place\_generator} -- принимает поле, возвращает точку на нем, в которой нет препятствия.

         \item \textit{action\_pool\_size} -- размер хранимого пула состояний и активностей, из которого из которого набираются данные для обучения сети.

         \item \textit{iters\_count} -- количество итераций обучения.

         \item \textit{batch\_size} -- размер пакета для кадой итерации обучения сети.
    \end{itemize}

    \item \textit{move}: перемещает животное в соответствии с выходом нейронной сети.

    \item \textit{get\_neurons}: возвращает результаты промежуточных слоев сети.

    \end{itemize}

\end{itemize}

Вспомогательные классы:

\begin{itemize}
    \item \textit{DefaultNetwork}

    Реализация нейронной сети по умолчанию. Имеет полносвязную архитектуру
с 2 скрытыми слоями по 100 нейронов. Получает на вход 16 чисел – расстояния до препятствий и
до выхода в каждом из 8 направлений.

    \item \textit{Shape}

    Базовый класс фигуры. Используется для построения поля.

    Методы:
    \begin{itemize}
        \item \textit{get\_distance}: возвращает расстояние до фигуры.
    \end{itemize}

    \item \textit{Circle}

    Класс окружности.

    \item \textit{Line}

    Класс прямой.

    \item \textit{Segment}

    Класс отрезка.
    
\end{itemize}

\end{document}